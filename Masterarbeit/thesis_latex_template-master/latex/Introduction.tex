\documentclass[thesis.tex]{subfiles}
\begin{document}

\chapter{Introduction}
\label{chap:introduction}
Earthquakes are still an ongoing thread for humans, infrastructure and buildings. There were some big earthquakes in the last years. (include photo) As technology improves early warning systems get more elaborate each year. Much work is put into researching earthquake types, earthquake physics simulation and the prediction of shaking or magnitude of an arriving earthquake. % An image right here at the top can look really cool!

\section{Background}
\subsection{Seismometers}
Since the 1900s seismographs were developed. Today's seismometers can measure ground motion very precisely. In earthquake rich countries there exit big networks of seismometers. We will also use such a network of seismometers in our dataset.
\subsection{Strength of an earthquake}
While we will have a look at the magnitude of an earthquake its important to keep in mind, that, while the magnitude is based on the physical properties of an earthquake, it is not the sole factor how strong the shaking is, or how much damage is going to occur.                                                                              
\section{Motivation}

\section{Goals} \label{bib:goals}
Algorithm goals: Safety before accuracy, detect big earthqaukes, even though they are underrepresented in data, estimate distance, detect, and estimate magnitude
Algorithmusentwurf, Ziele Schnelligkeit, Embeddedgeeignet, läuft auf schlechten Sensordaten, sagt Magnitude, Epizentrum mit einer gewissen Sicherheit richtig voraus
\section{Tasks}
Algorithm implementation, testing
\section{Data}
what kind of data, which structure, which dataset
\section{structure and contents of thesis}
explain how the thesis is going to explain the work
%Im nächsten Kapitel wird der verwendete Roboter beschrieben.
%Danach erfolgt die Beschreibung der entworfenen Algorithmen und ein Vergleich.

\subfilebib % Makes bibliography available when compiling as subfile
\end{document}